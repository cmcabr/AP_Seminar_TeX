% Document Metadata
\documentclass[12pt,letterpaper]{report}
\usepackage[utf8]{inputenc}
% Use for Arial Font \usepackage{helvet}
%  \renewcommand{\familydefault}{\sfdefault}
\usepackage{mathptmx, microtype}
\usepackage[none]{hyphenat}
\usepackage{tikz, textcomp, gensymb, graphicx, mathtools, amssymb, amsthm, hyperref, setspace}
  \hypersetup{
      colorlinks=true,
      linkcolor=blue,
      filecolor=magenta,
      urlcolor=blue,
      citecolor=black
      }
  \graphicspath{ {/home/lowebang/Pictures/} }
  \doublespacing
\usepackage[letterpaper]{geometry}
  \geometry{top=1in, bottom=1in, left=1in, right=1in}
\usepackage{fancyhdr}
  \pagestyle{fancy}
  \lhead{}
  \chead{}
  \rhead{\thepage}
  \lfoot{}
  \cfoot{}
  \rfoot{}
  \renewcommand{\headrulewidth}{1pt}
  \renewcommand{\footrulewidth}{1pt}
  \setlength\headsep{0.333in}
  \setlength\parindent{0.5in}
\usepackage{natbib}

% Command to Circle String
\newcommand*\circled[1]{\tikz[baseline=(char.base)]{
            \node[shape=circle,draw,inner sep=2pt] (char) {#1};}}

% Command to Set Oval Around String
\newcommand{\mymk}[1]{%
  \tikz[baseline=(char.base)]\node[anchor=south west, draw,rectangle, rounded corners, inner sep=2pt, minimum size=7mm,
  text height=2mm](char){\ensuremath{#1}} ;}

% Reference and Title Information
\title{The Plagues of Conformity on a Modern American Society \\
  \large AP Seminar - Individual Multimedia Presentation \\
  \large Individual Written Argument}
\author{\textit{"The Plagues of Conformity on a Modern American Society"} \\
        \large Word Count: 1819/2000}
\date{21 March 2022}

% Document
\begin{document}
  \maketitle
  \par Conformity is a vital part of the environment we as humans have created today. The study of human geography and sociology is built upon the study of different cultures, sharing common practices and aesthetics, conforming to the people around them. Even the smallest aspect of modern life such as the clothes we buy and the people that we communicate with contribute to our idea of social status, and to fall out of this norm is oft frowned upon. This perpetuates the debate on whether conformity is beneficial or harmful to civilization as a whole. However, based upon the effects of modern appearances, behavioral trends for authority, and the symbiosis of factions in politics, conformity can generally be considered harmful to civilization as a whole, while also being effective in niche groups.
  \par One of the first examples noted when in the discussion of conformity is that of aesthetics and appearance. In particular, dress is the first thing that can be seen when looking at different cultures around the globe. One major complication of such a conformity stems from the idea of fashion trends, and as \textit{The Atlantic} has noted, the trend of ``ultra-fast fashion''. This trend is defined by the rapid change in fashion sense common in the fashion industry and fashion retail, most commonly accelerated through internet advertising and sales \citep{RMonroe2021}. One of the first examples of a fashion boom started with the company Boohoo, who held their stores online rather than in-person, allowing for larger stocks of items and generally more ways to test new products. This internet-based economy did cause issues for the customers of the stores, however; specifically, the shoppers were becoming more pressured to keep with trends. Internet experiences tailored to users meant that stores that appealed to certain people could play to the person's insecurities even when they were not shopping for clothing; in one 2017 poll 41 percent of young adults did not want to copy an ``\#outfitoftheday'' because they felt pressured by backlash, causing money spent on clothing to skyrocket for some. This practice was heightened by the COVID--19 pandemic, when most people decided to begin shopping over the internet. Influencers who had owned these brands for years were able to peddle clothes manufactured overseas for cheap (sometimes with deadly or immoral practices such as child labor), all while brick-and-mortar stores closed their doors and allowed for internet stores to persist. Such a marketing strategy leads to stress on the general consumer. With clothing brands getting shoved in the face of customers over tailored internet posts, it was impossible to escape a certain brand if an algorithm deemed the individual enjoyed it. Social media stars and images of articles at cheap prices caused most to give in to the sales and buy cheap clothing that they did not need, believing that they were following a new or trendy brand that may die out in a few months. The broad brush of internet advertising is forced onto those who may have only seen a product once, and the algorithm functions like a digital parasite. For each person that stumbles upon this practice, social media forces conformity and shames those who do not fit into the mold. It stands to reason from this practice that conformity in dress may be a more harmful trait than previously anticipated, leading to the dissatisfaction of the consumer and a practice without an end goal.
  \par One of the more overlooked pieces of conformity is that of intergenerational relations. One major behavioral study conducted by the Department of Psychology and Neuroscience at the University of North Carolina Chapel Hill showed the effects of both peer pressure and parental influence on an adolescent mind. The study concluded that when shown positive and constructive behaviors, teen subjects listened to both peers and parents equally, unless parental behaviors conflicted with their own preconceptions; then, they listen to peer attitudes more \citep{KTDo2020}. This is the basis for what has commonly been refered to for the past half-century as the ``generational gap'', or the difference in social norms, political ideology, and general cultural behaviors between generations of people. As the aforementioned study details, this phenomenon is a natural sociological occurrence, resulting in a few key elements of modern culture. It seems that each generation has its own cultural influence and identities between them, causing in most cases tension between age demographics.
  \par Intergenerational conflict can not only be observed in everyday people, but also within literature shaped by the people it affects, such as in the example of Jhumpa Lahiri's novel, \textit{The Namesake}. Centered around the culture shock of an Indian family immigrating and settling into modern American life, one such excerpt focuses on one of the children, Gogol, as he feels outcast by the society around him. Despite his love of major English works such as the music of The Beatles and the literature of Adams and Tolkien, he is still pressured with gifts of academic or practical use by his family, including a series of short stories from the author he is named after, Nikolai Gogol. He loses interest in his family's customs and language, becoming more solemn and disinterested with the life around him. He is especially disgusted by his awkward name, not of American or Indian descent, but of Russian, and not a common Russian name at that. After inspiration from the author's past, however, Gogol considers a name change, to which his parents adamantly despise \citep{Lahiri2003}. While a tradition of passing on one's culture is seen as an honorable thing even into the modern age, forcing of these principles results in large tension between families and those who wish to assimilate into a newer culture. To force one to adhere to their ancestor's practices may lead to further straying from said practices.
  \par One major example in modern American society of a clash between ages is the relation between ``Generation X'' and ``Generation Y''. Because these terms are often not given solid definite time periods or definitions, we will define Generation X as anyone born between 1965 and 1982, and Generation Y as anyone born between 1983 and 2001. Generally speaking, Generation X was born in a time defined by great sociopolitical turmoil as the Cold War began to come to a peak and American corruption in politics was becoming more obvious and transparent to the general public. This led to a generation of individuals who turned against the ``establishment'', creating a more centered political environment which fought for both individual liberties and personal freedoms while also attempting to restore the nation to its former glory, leading to the era of Reagan and Bush \citep{Forbes2016}. This is the era in which Generation Y was born into; although an anti-government corruption message was being spread and the height of civil rights movements were shown in the public eye, such as support for the AIDS epidemic in homosexual communities and the second wave of feminism, race relations with authority and police were at a low point. In tandem with these issues, the aforementioned federal administrations were heavy on taxing and used less than sound economic policy, leading more people, and especially poorer minority groups, into poverty. This led to the current generation becoming more radical in both the fight for civil rights and worker equality, beginning a generation of moderate and extremist liberal tendencies \citep{BBC2017}. It is clear that as newer generations blossom and a culture grows in American society, the status quo may shift and a conforming to the previous generation will not stand in the interest of social justice and civil rights \citep{PEW2020}.
  \par The previous example of the issues of mental adherance does bring upon the discussion a valid point of contention -- conformity in political history. Since democratic principles were established and conflicting opinions were allowed to flourish in governments the world over, organizations and affiliations were founded with the sole intent of driving a specific ideology or goal to the forefront of the people's desires. Such methods were analyzed by Alexis de Tocqueville in his classic French text \textit{De La D\'{e}mocratie en Am\'{e}rique}. The works of Tocqueville center around the sphere of influence that the then-new American democratic system brought upon global society, especially in shaping European governments and North American social norms. Tocqueville notes that in the process of forming interest groups, Americans will constantly form associations that lean to a specific interest. Companies can be seen as a form of association, as it is pro-Capitalist and pro-worker in nature. Groups of careers may also be defined in Tocqueville's eyes as associations, as well as those of educational, religious, moral, and physical value. The thought is ended by noting that ``if it be proposed to advance some truth, or to foster some feeling by the encouragement of a great example, they form a society.'' \citep{deToc1835} de Tocqueville follows this point by addressing a defining trait of American political change: ``[Citizens]...fall into a state of incapacity, if they do not learn voluntarily to help each other''. His main argument forth is that a lone citizen of any democratic nation is weak in his power to change regulation for the benefit of his people, but an amalgamation of not only associations, but groups of associations, can be used to foster great change. Should this unity and conformity to political action and thought not occur, a larger political body such as government may rule over the people with no recourse. This argument may be seen as a defining American trait because it had been explored previously by Founding Fathers of American history, specifically within the works of James Madison. In an article for \textit{The New York Journal}, Madison anonymously argues that the aggregate interest of communities must be met by agreeance between multiple political factions; to form one large coalition sets political favor for the majority, the major flaw present in a true democratic society \citep{Madison1787}. An argument present from the exigence of the American dream, de Tocqueville's stance denotes the foundation of a strong democratic republic. It is the duty of the people, with disregard for difference in faction or need, to hold in their nature the good to help their fellow citizen and to produce for the common goal of the society.
  \par In conclusion, the idea of conformity is one that is commonly seen in modern life as an advantage to a group to maintain a form of status quo. In practice, however, forced conformity may be harmful to groups of people as a whole. Rather, what maintains healthy relations between groups of people in everyday life is the tolerance and mutual benefit between smaller social groups, which in turn conform in their own circles. Broad brushes on a general population may harm the population, but as the study of sociology has shown, it is the relation between smaller populations that maintains a homeostasis that we call diversity.
  \bibliographystyle{mla}
  \bibliography{research}
\end{document}